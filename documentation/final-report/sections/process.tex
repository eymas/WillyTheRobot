\section{Process}
This section details how the project was run, what methodology was used and how the quality of the separate parts of WTR were ensured.
At the end, there is also a short reflection about the processes as they were used during the project and what was learned from using them.

\subsection{Intial Intention}
\subsubsection{Methodology}
The chosen methodology for this project was SCRUM, since it is a methodology most of the group is familiar and comfortable with.
Another advantage that SCRUM offered for this project is that the scope can be flexible, since the exact goal of the project was not clear at the start.
Some parts of SCRUM were discarded, such as the daily meeting with the product owner, since the product owner would not have had time for that.

The sprints were set to be two weeks long, allowing for a total of 8 sprints.
This allowed the group enough time to focus in-depth on any single issue, without causing feature-creep \ref{trm::FC} in a sprint.
At the end of each sprint a sprint review was held with the product owner.
During this review, a small retrospective was also held to discuss sticking points and good performance.

\subsubsection{Weekly activities}
Every week, a meeting was held with either the product owner or the coach.
This turned out to be the same person, which was cause for apprehension at the beginning.
This fear turned out to be ungrounded, as Mischa Mol kept the two roles separate admirably.

\subsubsection{Process Overview}
Trello was chosen as a digital Scrum board.
While it would have been possible to organize a physical board, the environmental impact of a digital board is lower, and it is easier to pass on to new groups so that they can create an inventory of possible left-over tasks.
The board can be found \href{https://trello.com/willytherobot/home}{here}.
Several boards were created, though not all of them used.
The main bulk of the content is located in Project Tasks, with additional information found in the other board, TODO.

\subsubsection{Quality Management}
Several choices were made to ensure a consistent quality level.
First of all, all documents and code have been written in English, so that any international students can also work on this project without having to rely on google translate or such programs.
Secondly, a DOD \ref{trm::DOD} was made, which can be found in the project plan.
Some examples from the DOD are:
\begin{itemize}
\item Relevant documentation must be up to date, and checked by at least 1 reviewer
\item All code is uploaded to their proper branch, and checked by at least 1 reviewer
\item If changes have been made to the configuration or other parts of the code, update relevant sections
\end{itemize}
Additionally, in order to keep up to date with documentation and integrate with the pre-existing repositories, Git was used to ensure proper version control.
Every bit of code had to be reviewed by at least 1 person who didn't write the code.
This process was not necessarily enforced through pull request limitations, but rather through asking other members of the team to help out or check to see if the program functioned as intended.

\subsection{Correct Decisions?}
It is impossible to consider every aspect and consequence of a decision and still get work done, so sometimes a choice or decision has unintended effects. 
For example, despite the Git being organised into branches, due to a misunderstanding this was ignored a few times causing documentation to be in the wrong branch.
Does having to fix this and waste time on what is essentially an administrative error make using Git a mistake?

Choosing SCRUM could be considered a mistake, as the nature of the project, which already has an existing base and product, means that the flexibility of SCRUM is negated by the inflexibility of the product.
However, since the group already has experience in SCRUM makes the planning easier, as tasks are clearly split up into single tasks.

Git was useful, despite having caused the issues mentioned before.
The ability to easily keep code and documentation organized and up to date is worth the hassle it caused.
The reverting features of Git are also great, since several times they prevented having to redo entire sections due to over-eager deleting of documents.

\newpage
