\section{Security}
There are essentially two aspects to the security of WTR.
The first is the risk a malicious third party could pose to WTR if they wanted to abuse it.
The second is the risk WTR poses to the world around it.

\subsection{Vulnerabilities}
Ideally, WTR should be an air-gapped robot, with no possibility of wireless access, since it is going to be driving around at public events.
If WTR is broadcasting a network, then it would be possible to enter that and start commanding WTR to drive into people, for example.
Since this is unfortunately not possible as several social interaction features require an internet connection to some degree.

It would be advisable, however, to ensure that WTR gets an upgrade to its passwords.
Unfortunately, this group did not have the time to do this, but many of the passwords are vulnerable to botnets or other malicious attacks that exploit default passwords.
Many passwords are very weak, and could do with some updates.

Another vulnerability is that if someone has physical access to the switch, they could start spoofing messages and causing WTR to either crash or start behaving anomalously.
In order to prevent this, a small cover or physical protection could be used, such as a plexiglass cover that allows the internals of WTR to remain visible while preventing undesired physical access.

\subsection{Danger to Others}
At the moment, WTR is a lot less dangerous to the world around it than it was at the start of the project.
Previously, it had several sharp edges that ended up costing a member of the team a pair of jeans, so the issue had to be taken care of.

Since then, it has had covers placed over all sharp edges that could hit someone while WTR is moving.
There are still some sharp bits, but due to a lack of angle grinder this group was unable to fix this.

The batteries are still only kept in place due to their weight, so a sudden stop due to a collision or abrupt stop signal could shift the batteries, potentially snapping cables or such.
As is noted in the advice document, this needs updating.
Another point mentioned is the exposed parts, which while not life-threatening as the batteries only run at 24 volt, are still not exactly good to have exposed.
Again, refer to the advice document for the group opinion on how to solve these issues.

\newpage
