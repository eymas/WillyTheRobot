\section{Introduction}
At the start of the minor Future Technology in 2019 Q1\&Q2, students were allowed to choose from several projects.
Among those choices was this project, known as Willy the Robot, or WTR [\ref{trm::WTR}] for short.
WTR is a long running project, with 6 groups as of this one having worked on the machine.
The goal is to create an autonomously moving robot that can provide information to visitors during public events, such as Winnovation [\ref{trm::Winnovation}].

The goal of the project was to create a robot which could drive around safely at the aforementioned event.
WTR turned out to be a fairly difficult project, as the robot is already quite expansive and cumbersome, both in terms of hardware and software.
Many parts have been made separately, and although they are mostly modular, all the moving parts do need to work together to create the cohesive whole.
Many choices made during this project reflect this modular nature, again, incorporating both the hardware and the software.

This document details the personal experiences of the group and its members, and as such is of a less serious tone than the technical design and hardware design.
The first section details our personal reflections, such as what our role was, how well we thought the team worked together, whereas the second section is our reflection on the project as a group.s

\newpage
