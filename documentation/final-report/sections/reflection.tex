\section{Reflections}
\subsection{Jeroen van 't Hul}

\newpage

\subsection{Tom van den Noort}
At first, I was excited to start on this project considering the mixture between hardware and software. 
This excitement had dwindled after our first in-depth examination of WTR. 
A lot of improvements needed to be done and as such; we were slightly worried if we could achieve most of these, but we've exceeded our expectations.
Integrating, Improving, swapping new software plugins into ROS. We've gone through a few options to discover how simple our additions were, but we were met with challenges throughout.

\subsubsection{What went well?}
The start had not been the greatest, with the existing documentation holding some incomplete parts, we learned how to start up some software plugins i.e. RViz, but how do you actually use and configure them? 
Despite this, I've personally had no difficulty with absorbing the required information from the ROS wiki and other manuals to know the ins and outs of ROS itself. 
A notable achievement had been my investigation into the local planner; that would design paths for the robot to follow.
I've found that the one in the start of the project would just plan \textit{one} path and no other, which meant that if that path were to be blocked; the robot wouldn't know how to get past it, and spin in place, hoping the obstacle would be gone.
On the wiki page of this local planner I had found references to other planners, and after asking myself: ``What if I used one of those alternatives?'' I've looked at how to implement one of those, and I was surprised at how simple the swap was. 
More importantly; I was amazed by the great improvement of the new Timed Elastic Band Local Planner: It would dynamically recalculate the fastest path if an obstacle was thrown in front of it!

\subsubsection{What didn't go well?}
There was a downside to fiddling around with the software plugins and configuration files.
I did keep track of either what values I would change nor the original state of those, meaning that if the plugin were to behave erratically or cease function all together; I had nothing to turn back to if it worked well before my meddling. 
Although we did recover from such situations whenever they occurred, I did wish I would make a back-up of the files in question before changing them.
Even if that will imply that there'd be an incremental back up every few minutes.
Similarly, whenever there would be an issue we cannot resolve by ourselves, we lost the motivation to continue and wound up procrastinating. 
This happened with the (re)integration of the IMU sensor. 
We had already loathed the time wasted upon compiling the code and to discover that it would still function incorrectly made the stockpile even bigger, to the point we opted to take a break.
In the end, it took us around four or more weeks to get the IMU to function and integrated within ROS.

\subsubsection{What comes along for the ride?}
It was a delight to work with students from other directions than Software Engineering or Business IT \& Management. 
The knowledge we've shared had certainly helped us along the way with this project, including the usage of tools as \LaTeX, which Thomas has explained very well. Jeroen shared his mechanical designs and knowledge, notably of 3D printing and sensor diagnostics and was more than excited to do so.
The three of us joined forces in the project and have learned much about algorithms, maths, C++, Python and the publisher/subscriber model which ROS is based off of.
I myself have gotten interested in this model and might apply it in future projects.
3D Printing is exciting, with plenty of possibilities, if it so happens that I would get involved with a project for Internet of Things devices, the skills I've been taught would be of great help.
In terms of documentation, \LaTeX would save considerable time generating neatly designed reports without the instability of Microsoft Word templates.


\newpage

\subsection{Thomas Zwaanswijk}
This project has been a mixed bag, in my honest opinion.
Some parts went very well, such as integrating new data into ROS, or the accidental discoveries that made the project what it is today.
Naturally, it wouldn't be a mixed bag if there were only positives, so to provide the counterpoint the IMU had to be very difficult to get right.
I did end up learning a lot from this project, ranging from picking up where others left off, to complex mathematics I would never have thought I would use.

\subsubsection{What Went Well?}
Personally, I'm very happy with how the appearance of Willy has been improved.
It went from being a jumble of cables leading to hard-to-access Raspberry Pi's to something that resembles a decently organized machine.
Obviously, there is still room for improvement, but the simpler setup WTR has now really does allow for easier maintenance and upgrades.
Removing the very wasteful massive cables has been great to do, as well as placing every device in an accessible location to allow easy maintenance.

Another great point is that the 3D-printing adds to the concept of future technology very neatly.
A lot of the improvement in the visual aspect of WTR comes from the fact that the previous layers of tape have been replaced by neat boxes containing all the sensors and components of WTR.
I'm very happy with how some of those turned out, as at the beginning of WTR I did not have a lot of experience with 3D-modelling in Solidworks, but I learned a lot of neat new tricks throughout this project.

\subsubsection{What Went Less Well?}
But to the good must eventually come the bad, and in my case this is the IMU.
I must have spent at least a week trying to understand the mathematical principles behind quaternions, though that did pay off in the end.
Updating the code to work with the new MPU9250 was a nightmare.
Having to work on code made by people who think \code{TeaPotPacket[]} is a descriptive variable name is a painful process, to say the least.
Adding to this were the complexities that showed up when attempting to transmit floats on a serial communication system, which can only handle 8-bit integers.
Eventually I got through it all, but I did waste a lot of time, which is my biggest regret of the project.

I also ended up procrastinating a fair bit when code wouldn't work or I ran into issues I couldn't solve.
Personally, I find it very demoralizing to have worked on some code for several hours only to find it's essentially useless because a minor technicality prevents it from working.
I suppose this is also why it took me so long to get the IMU working.
I have a bad habit of putting off the hard tasks until last, so I can at least have the feeling of having done something by finishing up some easy tasks.
I did suppress this habit later on in the project, because I started feeling the pressure of the time limit approaching and I didn't have any choice but to get things done.

\subsubsection{Multi-Discipline Work}
It was also nice to work in a multi-disciplinary team, despite it also producing some unique challenges.
I had to push quite a lot to get the group to adopt the use of \LaTeX, but in the end I'm very glad that I did since it makes keeping documentation up to date a lot easier.
It was also nice to learn about some new ways of dealing with data from Jeroen, such as moving averages and feedback loops.
Having never worked with these before, I ended up learning a fair bit about how to implement those systems and what benefits they offer.
Another nice part of working in this team was that everybody had something to contribute.
Tom was great at working with ROS, RVIZ and Linux, which probably would have taken me half the project to get to grips with.
Jeroen, as I mentioned earlier, was great at working with the mechanical aspects, and his 3D-designs skills are top notch, which ended up teaching me a lot.
The downside of this would be that I ended up learning less about the new systems than I would have if I had to work on them myself.
But then again, if I'd had to do all this by myself, I would not even have gotten a third of the work done, so I'm quite happy with the end result.



\newpage
