\section{IMU}
The previous odometry measuring device, the MPU6250 \cite{mpu6250} was broken, which is why a new, updated version is currently being used.
Another reason a new version is used is that the MPU9250 \cite{MPU9250} also contains an AK8963, which is a 3-axis magnetometer.
This allow WTR to more accurately track orientation, since it can use the magnetometer to find its position relative to the north.
The data is collected as explained in this section \ref{sec::collect}, and then transmitted to a Raspberry Pi which deals with all the sensors.
After that has processed the data into a ROS standard message, it sends the data on to the topic, where it can then be used by planners.

\subsection{Collecting data} \label{sec::collect}
The main controller for collecting the data is an Arduino Uno, which uses I$^{2}$C for communication with the MPU9250.
This happens at a rate of \todo{find rate}.
The Arduino takes the raw data, and uses \href{https://github.com/sparkfun/SparkFun_MPU-9250_Breakout_Arduino_Library}{this library} to process the data.
This is needed because the MPU does not output quaternions by default.
This library allows the Arduino to calculate those automatically, and then send them on to the Pi.

\subsection{Arduino to Pi protocol}
The order in which the data is sent is naturally important.
The raspberry Pi expects the data packet to start with "\$ 0x03", and then the data of the quaternion, the gyroscope, the accelerometer, the temperature and lastly the magnetometer. 

Most of the program used to interface with the MPU9250 is based on a program by sparkfun, \cite{sparkfunMPU}, which is in turn based on the code by KrisWiner on Github \cite{kriswiner}.
Several parts have been stripped out, since no LCD is used.
Additionally, any debug messages have been removed after confirming that the code runs consistently.

Each of the sensors returns a 16 bit piece of data, as a signed 16-bit integer.
The exception to this rule is the quaternion calculation.
For that, a float with the values ranging between -1 and +1 can be expected.
Quaternions are rather tricky to work with, but more explanation can be found \href{https://www.3dgep.com/understanding-quaternions/#The_Complex_Plane}{here.}
As a quick warning, this is rather complex math, using both imaginary numbers and matrices.
It is worthwhile to study, since they form an integral part of the message to ROS.

As serial communications are used, the data cannot be transmitted as is.
In order to combat this, the data is split into two parts, with the quaternion again being an exception.
The two 8-bit unsigned integers are then transmitted, with the 8 bits representing the MSB \ref{trm::MSB} first and the LSB \ref{LSB} last.
These two integers are then spliced back into a signed 16-bit integer once the Pi has received them.


\begin{figure}[H]
    \begin{lstlisting}[language=c++,firstnumber=0]
        [ start byte ] = '$'
        [ start byte ] = 0x03
        [ quaternion value W ] = quaternion value from -1 to +1, multiplied by 100
        [ quaternion value X ] = See above
        [ quaternion value Y ] = See above
        [ quaternion value Z ] = See above
        [ accelerometer X raw data high byte \footnote{as in, the actual unprocessed data from the register of the MPU9250, not in m/s^{2}} ]
        [ accelerometer X raw data low byte ]
        [ accelerometer Y raw data high byte ]
        [ accelerometer Y raw data low byte ]
        [ accelerometer Z raw data high byte ]
        [ accelerometer Z raw data low byte ]
        [ gyroscope X raw data high byte ]
        [ gyroscope X raw data low byte ]
        [ gyroscope Y raw data high byte ]
        [ gyroscope Y raw data low byte ]
        [ gyroscope Z raw data high byte ]
        [ gyroscope Z raw data low byte ]
        [ magnetometer X raw data high byte ]
        [ magnetometer X raw data low byte ]
        [ magnetometer Y raw data high byte ]
        [ magnetometer Y raw data low byte ]
        [ magnetometer Z raw data high byte ]
        [ magnetometer Z raw data low byte ]
    \end{lstlisting} 
\caption{The order the Pi expects the data it receives to have}
\label{fig::dataformat}
\end{figure}


        
        
        