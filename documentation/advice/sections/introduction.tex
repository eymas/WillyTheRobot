\section{Introduction}
\label{sec::intro}

After having worked on WTR [\ref{trm::wtr}] for several months, the group has naturally learned a fair bit about how it works and more precisely, what doesn't.

This document is aimed at both the product owner and any groups or teams that might work on WTR after this group.

WTR is a curious machine which has a long and storied history.
It started of as a garbage collection robot, but after a while the plan changed, as plans always do.
To paraphrase Heltmuth von Molke, no plan survives contact with reality.
When the project came into the possession of Windesheim, the goal shifted from a \textit{Holle Bolle Gijs} that could drive autonomously to a mobile information centre.
To this end, the target of the 2019 Q1\&Q2 group was to increase its autonomous capabilities.

As of the end of the project, we as a team are quite satisfied with the current capabilities, but this is not to say the project is finished.
The \href{https://windesheim-willy.github.io/WillyWiki/}{Wiki} has an accurate table displaying what is left to be done.
According to that, all it needs is an improvement in the social interaction category.
The opinion of the group is that while this is true, it also needs a thorough cleaning, as well as some work on the exterior casing.

This document will additionally contain advice on the way to work with WTR, gathered through experience and annoyance.
\newpage 