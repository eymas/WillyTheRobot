\section{Conlusion}
In short, there is a lot to improve about WTR.
By creating this document, an inventory has been created about the deficits WTR has, how they could potentially be solved, and why the solutions that were chosen are the best available.
The list is as follows:

\begin{tabular}{|C{3cm}|L{6cm}|L{6cm}|}
\hline
\textbf{Issue}                 &\textbf{Solution} & \textbf{Reasoning} \\ \hline
Front wheels out of sync       &Rotary Encoders on a flexible mount    & By keeping track of the amount of rotations, slipping and friction can be countered to some degree \\ \hline
Failing wooden frame           & Reinforce with metal mounts    & Prevent WTR from falling apart, and a metal mount is more reliable that glue and tape \\ \hline
Shifting batteries    & Creating a frame or wedges to keep the batteries in place & Electronics are sensitive enough as is, and don't need more jostling than is absolutely necessary \\ \hline
Power Converter box is unsafe & Creating a 3D-printed box with increased air-flow & Electronics can be sensitive to heat, and the current solution has next to no cooling efficiency \\ \hline
Rear wheels pivot    & Accepting this as an issue and working to minimize the impact & Any solution that would have notable impact would require such extreme amounts of reworking it would become pointless \\ \hline
Straightening out front wheel & Adjusting suspension so the wheel won't grind against the frame & After a collision, the wheel was jammed against the frame, impacting the driving capabilities. \\ \hline
Recovery Behaviours & Creating custom plugins for new behaviours to overwrite old ones & Current behaviours do not take WTRs shape into account and as such need to be replaced \\ \hline
\end{tabular}

These issues are generalized, and each will need to be broken up into sub-tasks.
For example, simply adding the rotary encoders will not be enough.
There needs to be a complete model integrated into the motor-controller in order to ensure that they can properly be utilized.
While some of these improvements are small, they add up to a safer robot that will be able to more accurately move where and when needed.

\newpage