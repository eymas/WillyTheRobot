\section{Jeroen van 't Hul}
\subsubsection{Introduction}
As being a mechatronics student from Saxion following the Minor Future Technology I tried to apply my knowledge of mechatronics into our project, and share it with my team members. 
My role in the project group is being the project leader, this mainly includes keeping in contact with various stakeholders and other members of Future Technology. 
As for the project itself the role is less visible as my team members keep initiative very well. 
In this reflection I will discuss the results I've achieved and make a conclusion afterwards.


\subsubsection{Results}
Throughout the project I've worked on various beneficial parts. 
Mechanically I've improved the wheel mounting/suspension of the robot, designed and printed various mountings for sonar sensors, the lidar sensor, rotary encoders. 
Electrically I designed a wire harness for the rotary encoders and for the sonar sensors. On the programming side, the motor controller code is drastically changed to read out rotary encoders, two PID loops and one controller. This is also documented by means of self-explanatory flow charts and tables. 
In order to check the control loop and tune the PID parameters I designed a viewer with charts, which also turned out to be very useful in understanding the drive and turn messages from ROS, and tuning accordingly.
I also spent a lot of time in tuning parameters for the AMCL. This unfortunately did not gave the appropriate results. 
For the 9 axis IMU I also designed a 3D viewer to get a better understanding on the output of the sensors.
Currently I am working on the sonar sensors, at the moment ROS is able to read out the sensors.

\subsubsection{Conclusion}
Being a multidisciplined project I've learned a lot during this project, also from my other team members.
From them I got a basic understanding of GitHub, learned to document with latex, got insight in ROS and Linux. 
Personal 3D printing and CAD drawing skills have improved. 
I also had to heavily apply knowledge from control engineering to be able to get the robot to drive smoothly with the control loop.
On the other hand I failed to fix the drift issue in the AMCL on which I spent a lot of time, this was a bit demotivating, and this is something to look out for personally. 
