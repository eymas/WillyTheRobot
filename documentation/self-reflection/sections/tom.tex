\section{Tom van den Noort}

As a Software Engineer, I've delved head first into the rabbit hole of the Robotic Operating System (henceforth ROS), absorbing the information about how it is built, how its components function and how to manipulate them to do what I and others want it to. 
The first mistake I made is not to document this journey, although I did share my discoveries in presentations and explanations to my team-mates and other stakeholders. 
The available information on the Willy Wiki, along with what I've picked up along the way have helped us in getting started on optimizing the many parts and algorithms that make the robot navigate autonomously. With dedicated time, we would contribute our discoveries upon the Wiki to make sure that future groups can use the information we've gathered for their own progress. 
ROS runs primarily on the Linux operating system, as do the Raspberry Pi's that are used on the robot, my team-mates didn't quite know how Linux worked, as such I've taken steps to simplify the usage of Linux (by implementing command aliases and scripts) as well as to explain how they can use the Command Line Interface (CLI), I intend to add this information unto the Wiki as well. 