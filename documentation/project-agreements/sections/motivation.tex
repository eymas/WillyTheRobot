\section{Motivation}
This chapter describes the motivation for the project. 
it described why it started, what the current problem is and the set goal to reach within the allotted time.

\subsection{Context}
Windesheim University wants to have a robot which can drive autonomously and interact with guests to provide them with information on open days.
Several project groups have been working on this goal prior to this project, creating and/or completing various modules that work together to realize the autonomy and interactivity of the robot.
The goal that this group wants to reach will be described below and continues from what the previous group(s) have left behind.

\subsection{Problem}
The main problem of the robot in its current state is the autonomous driving capabilities. 
The robot is able to drive autonomously to a certain degree, but it lacks a fluent and safe navigation system. 
Next to this the overall impression of the robot is messy. 
The cables are not properly managed, sensors and controllers are loosely attached. 
Mechanically the undercarriage is not stable and one wheel rubs against the frame. 
These and other flaws pose a potential safety risk to the robot and others nearby. 

\subsection{Goal}
Out of the problems stated above, autonomous driving has been chosen as the core focus of this project. 
The challenge is to improve upon the driving system(s) to make sure the robot will be able to drive autonomously without posing a threat to itself, the environment or others.
Currently, the scope is limited to T5 in Windesheim, but the ultimate end-goal is that Willy can be used anywhere without issue.
Additionally, the documentation of the project itself needs to be corrected and maintained, giving future groups the ability to pick up and continue the project without any delay or misunderstandings.

\subsection{History}
This is not the first project group to work on Willy.
At this point in time, there have been 4 other groups to work on the robot.
That means that there is a lot of pre-existing code and functionality, all of which must be considered and either re-worked or worked with.
Another curiosity is that Willy started off as a completely different product.
Originally, it started as a garbage collection robot, but somewhere during development of Willy the company responsible for Willy handed it over to Windesheim, who chose to turn Willy into a robot greeter to be used during public events.
As such, the choice was made to focus on autonomous driving, with as a sub-goal to reduce the clutter present on Willy and improve the way the sensors are mounted.

\subsection{Parties, Roles and Stakeholders}
Currently, the stakeholders are as follows:
\begin{center}
\begin{tabular}{|l|l|l|}
\hline
\textbf{Name} & \textbf{Role} & \textbf{Contact Info} \\ \hline
Mischa Mol 	& Product Owner/Coach & \href{mailto:mc.mol@windesheim.nl}{mc.mol@windehseim.nl} \\ \hline
Thomas Zwaanswijk & Project member & \href{mailto:thomas.zwaanswijk@windesheim.nl}{thomas.zwaanswijk@windesheim.nl} \\ \hline
Tom van der Noort & Project member & \href{mailto:tom.vanden.noort@windesheimflevoland.nl}{tom.vanden.noort@windesheimflevoland.nl} \\ \hline
Jeroen van `t Hul & Project leader & \href{jeroen.van.t.hul@windesheim.nl}{jeroen.van.t.hul@windesheim.nl} \\ \hline
\end{tabular}
\end{center}
\newpage
