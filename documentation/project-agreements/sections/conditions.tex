\section{Methodology}
For this project, an agile methodology is very convenient.
There are of course multiple options when it comes to working in an agile manner, but as the entire group already has experience with SCRUM, that will be the method used.
The sprint duration will be two weeks, with daily stand-ups  being held every day.

\subsubsection{Daily Stand-up}
\label{sec::standup}
Every morning as soon as every member is present a stand-up will be held.
During this session every member tells the others what tasks they worked on yesterday, what tasks are going to be done today, as well as any anticipated challenges.
The purpose of this meeting is to ensure every member of the group knows what the others are doing, and allow them to offer accurate advice on current issues.

\subsubsection{Sprint review}
\label{sec::sprintrev}
A sprint review will be held at the end of every sprint, which means that this will be a bi-weekly event.
This should include a demonstration or presentation about the events that occurred in the last two weeks.
The product owner should be invited to every review, so that feedback from the product owner can be incorporated into the next sprint planning.

\subsubsection{Sprint Retrospective}
The retrospective offers the group a chance to consider what went well and what went poorly each sprint.
Therefore, this should be held after the review, but before the planning where possible.
The retrospective should be kept short and simple as much as possible, since after it the planning needs to be held.
The retrospective will be done through Trello, creating lists of what went well, what went poorly, and what helped along the way.
It ends when all cards have been discussed and any issues brought up have been resolved.

\subsubsection{Sprint Planning}
After every review, a sprint planning is held.
The goal of this is to select the tasks which have priority and move them to the appropriate sprint backlog. 
Any work that has not been finished is moved back to the backlog, and re-evaluated.
Any sprint planning beyond the first should start after the retrospective is done, preferably around noon, and finish at 4 o'clock.
After every feature has been discussed and supplied with story points, the planning is finished.
\newpage

\subsection{Daily \& Weekly Activities}
Certain activities are held on a regular basis.
These include the agile standards, weekly meetings and reports, as well as daily gatherings.

\subsubsection{Daily Stand-Up}
As has been mentioned before, a stand-up will be held every day.
They are described in this section \ref{sec::standup}
\subsubsection{Weekly Meeting With Coach/Product Owner}
Especially at the beginning, it is important to meet with the coach Mischa Mol quite often, as a lot of work has already been done by previous teams.
In order to ensure that the main goal of the project is not missed, every week progress and difficulties can be discussed.
This allows feedback from the coach in addition to that given during the sprint review \ref{sec::sprintrev}.

\subsection{Quality Management}
In order to ensure quality in both code and documentation, several measures have to be taken.

One of these is having to have at least one person review any code and documentation before it can be merged into a development branch.
This means that errors are more likely to be caught, since they are generally more obvious to people who did not write a section.

Another method that will be used is code standards. 
These can be found in appendix ~\ref{app::codecon}.

Code testing must be done by at least one person, who also then writes down the test results.
This allows for better transparency in what went wrong and why, resulting in easier bug-fixing.
\newpage

\subsection{Tools}
Not every tool will be supported in this project, as a lot of decisions have already been made by previous teams.

\begin{itemize}
\item Whatsapp - Communication within the team
\item GitHub - VC remote and reviews
\item Trello - Planning boards and reviews
\item Google Calendar  - Keeping track of meetings and planned absences
\item Robotic Operating System - The system used to control Willy, and what must be programmed in
\item UMLet - Free and open source tool for drawing diagrams, works well with GitHub for version control
\item Git - version control
\item MikTex - \LaTeX distribution for Windows
\item \LaTeX editor of choice - Note that only one member of the team has \LaTeX experience, and only in TexWorks
\item Google Drive - file sharing and places for keeping quick notes
\item Skylab - control of visuals of robot
\end{itemize}

\subsection{Agreements}
\begin{itemize}
	\item All code must follow the Google Style where it is relevant to do so.
	\item Branches can only be merged into development after at least one other member of the group has performed a code review and marked it as passing on Git.
	\item Code version control is done through GitHub
	\item Any work submitted must follow the Definition Of Done, or DoD.
\item General work days are from 9 to 4 where possible, with exception on mondays where PV subjects are concerned
\item The target amount of hours per week is 32 hours, regardless of their distribution throughout the week.
\item In case of late arrival, should there be unavoidable circumstances it can be ignored
\item Every time a group member is late due to their own fault, they must buy coffee for all group members.
\item After every third time, they must buy a treat of some description for the entire project group as well.
\item Communication is primarily done through Whatsapp or in person when available.
\end{itemize}
\newpage
