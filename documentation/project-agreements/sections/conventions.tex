\section{Conventions}
\subsection{Code Conventions}
\label{app::codecon}
Code conventions allow for more readable code.
By having every member follow the same standard, it prevents confusion in parts such as naming schemes.
The choice has been made to follow the Google C++ style conventions, which can be found at \href{https://google.github.io/styleguide/cppguide.html}{\cite{cppstyle}}.

\subsection{Document Conventions}
Conventions for \LaTeX documents are as follows:
\begin{itemize}
\item Since \code{\textbackslash\textbackslash} is used to end the current line, it is reasonable to do so in the source file as well, but use it only when needed.
\item If referencing sources, create a BibTex entry in the .bib file (check \href{https://en.wikibooks.org/wiki/LaTeX/Bibliography_Management#A_few_additional_examples}{this link} for examples).
\item When referencing an internal section or paragraph, use \textbackslash \code{label\{\}}, with a short abbreviation to note the type of section referenced. E.g. \code{\textbackslash label\{sec::conventions\}}.
\item Citations are done in IEEE format, done automatically by \LaTeX.
\item Labels are put under every image or figure for ease of reference.
\item New sentences start on a new line.
\item One root file will contain all sections as a final document.
\item Section sources should be in the proper folder, \code{./sections}
\item Images and figures should be placed in \code{./images}
\item The root file (typesetting target) and the \code{.bib} file should be the only files in the root directory.
\item Document names are not capitalized, and a hyphen is used rather than a space.
\end{itemize}

\subsection{Version Control Conventions}
The following conventions are to be used when working with Git/GitHub.
\begin{itemize}
\item Branch names should follow this format: \code{subject/name-of-feature}, e.g. \code{doc/project-plan} or \code{code/motor-control}.
\item \code{master} and \code{development} should be protected from pushes. Pull requests need to be reviewed by colleagues that did not work on functionality or branch as much as possible.
\item \code{master} should only ever be updated by a pull request from \code{development}
\item Commit messages should be clear in purpose and necessity.
\item Commits should be kept neutral, no personal attacks or callouts.
\item Code review must be kept equally neutral and impersonal, focus on the content, not the person who made it.
\end{itemize}
\newpage
